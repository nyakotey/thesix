% =============================================================================
% CHAPTER 2: LITERATURE REVIEW
% =============================================================================

\chapter{Literature Review}

\section{Introduction}

This chapter reviews the relevant literature related to your research topic. The literature review should be comprehensive and critical, identifying gaps that your research will address.

\section{Theoretical Framework}

Discuss the theoretical foundations that underpin your research. This section should establish the theoretical lens through which your research will be analyzed

\section{Key Concepts and Definitions}

Define important terms and concepts that are central to your research.

\section{Previous Studies}

Review previous studies related to your topic. Use Harvard referencing style as required. For example:

According to \citet{Cobbinah2015a}, sustainable development must be reconsidered in the context of poverty and urbanisation in developing countries.

Multiple authors can be cited as \citep{Duah2015, PokuBoansi2015} when discussing the urban poor and rent issues in Kumasi.

For newspaper articles, see \cite{Tamakloe2008}\footnote{Newspaper citations can provide valuable contemporary perspectives on research topics, though they should be used judiciously in academic work.}

The approach proposed by \cite{amendolaDroneLandingMoving2023} learns a single policy for multiple tasks, which is relevant for adaptive systems in robotics. 

\cite{azarDroneDeepReinforcement2021} present a comprehensive review of deep reinforcement learning applications in drone technology, highlighting the potential for autonomous navigation and control.


\section{Research Gaps}

Identify the gaps in existing literature that your research will address.

\section{Conceptual Framework}

Present your conceptual framework if applicable.

\section{Chapter Summary}

Summarize the key points from this literature review and how they relate to your research.
